\section{Введение}

\subsection{Цель работы}

Основной целью работы является ознакомление с принципами функционирования, построения и особенностями архитектуры суперскалярных конвейерных микропроцессоров.
Дополнительной целью работы является знакомство с принципами проектирования и верификации сложных цифровых устройств с использованием языка описания аппаратуры SystemVerilog и ПЛИС.

\subsection{Основные теоретические сведения}

RISC-V является открытым современным набором команд, который может использоваться для построения как микроконтроллеров, так и высокопроизводительных микропроцессоров.

В данной работе исследуется набор команд RV32I, который включает в себя основные команды 32-битной целочисленной арифметики кроме умножения и деления. 

Набор команд RV32I предполагает использование 32 регистров общего назначения x0-x31 размером в 32 бита каждый и регистр pc, хранящего адрес следующей команды. 
Все регистры общего назначения равноправны, в любой команде могут использоваться любые из регистров. Регистр pc не может использоваться в командах.

Архитектура RV32I предполагает плоское линейное 32-х битное адресное пространство.
Минимальной адресуемой единицей информации является 1 байт.
Используется порядок байтов от младшего к старшему (Little Endian), то есть, младший байт 32-х битного слова находится по младшему адресу (по смещению 0).
Отсутствует разделение на адресные пространства команд, данных и ввода-вывода.
Распределение областей памяти между различными устройствами (ОЗУ, ПЗУ, устройства ввода-вывода) определяется реализацией.

Архитектура RV32I, как и большая часть RISC-архитектур, предполагает разделение команд на команды доступа к памяти (чтение данных из памяти в регистр или  запись данных из регистра в память) и команды обработки данных в регистрах.